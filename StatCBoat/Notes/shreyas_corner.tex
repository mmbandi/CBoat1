\documentclass[aps, floatfix, superscriptaddress]{revtex4}
\usepackage{amsmath, amssymb, amsfonts, gensymb}
\bibliographystyle{apsrev}
\usepackage{graphicx}
\graphicspath{ {Figures/} }

\begin{document}
 
\title{Shreyas' corner - deriving the similarity solution}
\author{Shreyas Mandre}
\affiliation{School of Engineering, Brown University}
\maketitle

\begin{figure}
\includegraphics[width=8cm]{Figures/Theory_Schematic}
\caption{Schematic setup of the problem. A point source located at the origin releases insoluble surfactant on the interface of a semi-infinite pool of fluid. The interface is along the $z=0$ plane. The Marangoni stress on the interface arising from the non-uniform distribution of surfactant drives a flow.}
\label{fig:Theory_Schematic}
\end{figure}

\section{Mathematical model}
Consider an semi-infinite bath of fluid with a free interface along the $x-y$ plane as schematically shown in figure \ref{fig:Theory_Schematic}.
An insoluble surfactant that spreads adsorbed to the free interface is released at the rate $Q$ from a point source located at the origin.
The axisymmetric forcing suggests description of the flow in cylindrical polar coordinates $(r,z)$ using the radial and axial components of velocity $u(r,z)$ and $w(r,z)$ respectively, and the surfactant concentration by $c(r)$.

The fluid flow satisfies the incompressible Navier-Stokes equations
\begin{align}
 u u_r + w u_z &= -\dfrac{1}{\rho} p_r + \nu \left[ \dfrac{1}{r} \left( ru_r \right)_r -\dfrac{u}{r^2} + u_{zz} \right], \\
 u w_r + w w_z &= -\dfrac{1}{\rho} p_z + \nu \left[ \dfrac{1}{r} \left( rw_r \right)_r + w_{zz} \right], \\
 \dfrac{1}{r} (ru)_r + w_z &= 0.
\end{align}
The interface is assumed to be fixed at $z=0$ due to the presence of strong surface tension compared to other stresses that arise in the fluid (cite Jensen here?), which implies
\begin{align}
 w=0 \qquad \text{ at } \qquad z=0.
\end{align}
The velocity field is forced on the boundary by a Marangoni stress due to non-uniform distribution of the surface tension $\sigma$
\begin{align}
 \mu u_z = \sigma_r, \qquad \text{ where } \qquad \sigma = \sigma_0 - \Gamma c,
\end{align}
$\Gamma$ being the assumed constant rate of change of surface tension with surfactant concentration.
The surfactant is transported along the interface by advection as
\begin{align}
 (ruc)_r = 0 \qquad \text{ at } \qquad z=0; \qquad \text{ and therefore } ruc = q = \text{ constant},
\end{align}
where $q$ is the strength of the point source.
We assume the Peclet number to be large, as is the case for most surfactants, so that diffusion of the surfactant can be neglected.
Once the solution is found we derive quantitatively the conditions on the parameters that satisfy our assumptions.

These equations were solved in COMSOL using a combination of its Computational Fluid Dynamics and Mathematical Modelling modules. 

\section{Similarity solution}
The power law in the surface velocity profile implies scale free dynamics, and therefore a self-similar flow.
We attempt to find the flow profile here.

Using the boundary layer approximation, the governing equations simplify to:
\begin{align}
 \dfrac{1}{r} (ru)_r + w_z &= 0, \\
 uu_r + wu_z &= \nu u_{zz},
\end{align}
with the boundary conditions
\begin{align}
 ru(r,z=0) c(r) &= q = \text{constant}, \\
 \mu u_z(r, z=0) &= - \Gamma c_r.
\end{align}
These equations represent the conservation of mass, momentum, surfactant, and the Marangoni stress balance.

The two boundary conditions can be combined into one by eliminating $c(r)$ using
\begin{align}
c(r) = \dfrac{q}{r u(r,z=0)}, \text{ to get } \\
u_z(r, z=0) = -\dfrac{\Gamma q}{\mu} \left( \dfrac{1}{r u(r, z=0)} \right)_r = -K \left( \dfrac{1}{r u(r, z=0)} \right)_r,
\end{align}
where $\Gamma$, $q$ and $\mu$ always appear in the combination represented by $K$.

Stream function $\psi(r,z)$ can be used to satisfy mass conservation by defining:
\begin{align}
 u = \dfrac{1}{r} \psi_z, \quad w = -\dfrac{1}{r} \psi_r.
\end{align}

The self-similar flow can then be described using a similarity variable
\begin{align}
 \xi = \dfrac{z}{\delta(r)},
\end{align}
where $\delta(r)$ is the boundary layer thickness, and
\begin{align}
 \psi(r,z) = a \Psi(r) \delta(r) f(\xi),
\end{align}
where we expect $\delta(r)$ and $\Psi(r)$ to be power-laws in $r$. The following relations for the derivatives of these quantities are handy:
\begin{align}
 \dfrac{\partial \xi}{\partial z} = \dfrac{1}{\delta(r)}, \qquad \text{ and }  \qquad \dfrac{\partial \xi}{\partial r} = -\dfrac{\xi}{r} \left( \dfrac{r \delta'(r)}{\delta(r)} \right) = -\dfrac{m \xi}{r}. 
\end{align}
Note that since $\delta(r) \propto r^m$ is a power law, we expect $r\delta'(r)/\delta(r)$ to be the exponent $m$.
We similarly assume $\Psi(r) \propto r^n$ and expect $\Psi'(r) = n\Psi(r)/r$.
According to this ansatz, 
\begin{align}
 u(r,z) = \dfrac{a \Psi(r)}{r} f'(\xi), \qquad \text{ and } \qquad  
 w(r,z) =-\dfrac{a \Psi(r) \delta(r)}{r^2}  \left( (m+n) f(\xi) - {m\xi} f'(\xi)\right).
\end{align}
To substitute this ansatz in the momentum balance, we need the following expressions:
\begin{align}
 u_r = \dfrac{a\Psi(r)}{r^2} \left( (n-1) f'(\xi) - {m \xi} f''(\xi) \right), \qquad 
 u_z = \dfrac{a \Psi(r)}{r \delta(r)} f''(\xi), \qquad \text{ and } \qquad 
 u_{zz} = \dfrac{a \Psi(r)}{r \delta(r)^2} f'''(\xi)
\end{align}

The momentum balance then reads
\begin{align}
 \dfrac{a^2  \Psi(r)^2}{r^3} \left[ f'(\xi) ((n-1) f'(\xi) - {m \xi} f''(\xi) )  - f''(\xi) ( (m+n) f(\xi) - {m\xi} f'(\xi) ) \right] = \dfrac{\nu a \Psi(r) }{ r\delta(r)^2} f'''(\xi),
\end{align}
which simplifies to
\begin{align}
\delta(r) = \sqrt{\dfrac{\nu r^2}{a \Psi(r)} } \qquad \text{ i. e. } m = 1-\dfrac{n}{2} \qquad \text{ and } \qquad  (n-1) f'(\xi)^2  -  \dfrac{2+n}{2} f(\xi)f''(\xi) = f'''(\xi),
\end{align}
In order to substitute the ansatz in the boundary condition, we evaluate
\begin{align}
 \left( \dfrac{1}{r u(r, z)} \right)_r = \left( \dfrac{1}{a \Psi(r) f'(\xi)} \right)_r = -\left( \dfrac{1}{r a \Psi(r) f'(\xi)^2} \right) \left( n f'(\xi) - {m\xi} f''(\xi) \right)
\end{align}
which upon said substitution yields
\begin{align}
 \dfrac{a \Psi(r)}{r \delta(r)} f''(0) = Kn \left( \dfrac{1}{r a \Psi(r) f'(0)} \right).
\end{align}
Substituting $\delta(r)$ and equating the dimensional, the power law, and the self-similar variables separately yields
\begin{align}
a = K^{2/5} \nu^{1/5}, \qquad \Psi(r) = r^{2/5}, \qquad f''(0) f'(0) = n = \dfrac{2}{5}.
\end{align}

\vspace{3mm}
The solution so far may be summarized in terms of a self-similar profile $f(\xi)$ as:
\begin{align}
\begin{split}
 \psi(r,z) = (a \nu)^{1/2} r^{6/5} f(\xi), \text{ where } a=K^{2/5} \nu^{1/5}, \quad \xi = \dfrac{z}{\delta(r)}, \quad \delta(r) = \left( \dfrac{\nu }{a }\right)^{1/2} r^{4/5}, \\
 u(r,z) = ar^{-3/5} f'(\xi), \quad w(r,z) = -\dfrac{ (\nu a)^{1/2} }{5} r^{-4/5}  \left( 6f(\xi) - 4\xi f'(\xi)\right), \quad c(r) = \dfrac{q}{a r^{2/5} f'(0)}, \\
 n=\dfrac{2}{5}, \quad m = \dfrac{4}{5}.
\end{split}
\end{align}

The ordinary differential equation for $f(\xi)$
\begin{align}
 f'''(\xi) +  \dfrac{3}{5} f'(\xi)^2  +  \dfrac{6}{5} f(\xi)f''(\xi) = 0,
 % - \dfrac{8}{5} \xi f''(\xi) f'(\xi) 
\label{eqn:selfsimilarode}
\end{align}
is to be solved with the boundary conditions
\begin{align}
 f(0) = 0, \qquad f''(0) f'(0) = \dfrac{2}{5}, \qquad f'(-\infty) = 0.
\label{eqn:selfsimilarbc}
\end{align}

Equations (\ref{eqn:selfsimilarode}-\ref{eqn:selfsimilarbc}) are numerically solved using a shooting method, which we outline next. 
(The Python 3 program used to implement this method is shown in figure \ref{fig:python3program}).
The method starts with a guess for $f''(0)$ (represented as \texttt{fpp0} in the program) and sets the corresponding $f'(0)$ using \eqref{eqn:selfsimilarbc}.
The initial value problem with the guessed initial condition is then solved numerically using a fourth order Runge Kutta method and its asymptotic behavior as $\xi \to \infty$ is examined. The solution asymptotically either diverges to $\infty$ or $-\infty$ as $f\sim C \xi^{2/3}$, thereby violating the far-field boundary condition in \eqref{eqn:selfsimilarbc}. 
However, between the cases that diverge to $\infty$ and those that diverge to $-\infty$ is one solution that remains bounded.
For this case $f$ approaches a constant values, $f_\infty$.
The possibility of an asymptotic approach of $f$ to $f_\infty$ may be examined by making the ansatz $f = f_\infty + \epsilon g(\xi) + \dots$ for $\epsilon \ll 1$ and solving for $g$.
The resulting solution is
\begin{align}
 f(\xi) = f_\infty + \epsilon \left( a_1 + b_1 \xi + c_1 e^{-6f_\infty \xi/5}  \right) + O(\epsilon^2),
\end{align}
where $a_1$, $b_1$ and $c_1$ represent arbitrary constants of integration.
Note that $b_1$ must be zero so that asymptotic ordering of the solution is maintained as $\xi \to \infty$ and $a_1$ may be absorbed into $f_\infty$.
The objective of the shooting method is to guess the initial value $f''(0)$ such that this bounded solution with $b_1=0$ is asymptotically achieved.
We find this solution by successive bisection of the interval of $f''(0)$ with end points that lead to diverging solutions with opposite signs.
This bisection was implemented manually to determine that $f''(0) = 0.402287361293201$ solves (\ref{eqn:selfsimilarode}-\ref{eqn:selfsimilarbc}) numerically to 15 digit accuracy.
The resulting solution for $f(\xi)$ and its derivatives is shown in figure \ref{fig:similaritysoln}.

\begin{figure}
\begin{verbatim}
import numpy as np

def fun(t, y):
    f = y[0]
    fp = y[1]
    fpp = y[2]
    fppp = - 0.6*fp**2 - 1.2*f*fpp 
    return np.array([fp, fpp, fppp])
 
t = 0
N = 60000
dt = -1e-3
T = np.linspace(0, dt*N, N+1)
fpp0 = 0.402287361293201
y = np.array([0, 0.4/fpp0, fpp0])
Y = np.zeros((N+1,3));

ii = 0
for ii in range(N+1):
    Y[ii,:] = y[np.newaxis, :]
    dy1 = dt*fun(t, y)
    dy2 = dt*fun(t+dt/2, y+dy1/2)
    dy3 = dt*fun(t+dt/2, y+dy2/2)
    dy4 = dt*fun(t+dt, y+dy3)
    y = y + (dy1+2*dy2+2*dy3+dy4)/6
    t = t + dt

\end{verbatim}
\caption{Python3 program to solve the initial value problem for $f(\xi)$. The program converts the thrid order ordinary differential equation \eqref{eqn:selfsimilarode} to three first order equations for the variables \texttt{y[0]}, \texttt{y[1]}, and \texttt{y[2]}, and integrates them using the standard fourth order Runge-Kutta method.}
\label{fig:python3program}
\end{figure}

\begin{figure}
\includegraphics{Figures/bl_profiles}
\caption{The solution of (\ref{eqn:selfsimilarode}-\ref{eqn:selfsimilarbc}) obtained using shooting method.} 
\label{fig:similaritysoln}
\end{figure}

\end{document}
