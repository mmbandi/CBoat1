\pdfminorversion=4
\documentclass[compress,10pt,dvipsnames,svgnames,pdftex]{beamer}
\usepackage[utf8]{inputenc}
% \usepackage{default}
% \useoutertheme{split}
% \useinnertheme{rounded}
% \usecolortheme{seahorse}
% \usecolortheme{lily}

\usepackage{pgf,xcolor}
\usepackage{amsmath,amsbsy,amssymb}
\usepackage{multimedia}
\usepackage[framemethod=TikZ]{mdframed}

\mdfdefinestyle{MyFrame}{%
    linecolor=black!20!white,
    outerlinewidth=0.5pt,
    roundcorner=5pt,
    innertopmargin=0.5\baselineskip,
    innerbottommargin=0.25\baselineskip,
    innerrightmargin=2pt,
    innerleftmargin=5pt,
    backgroundcolor=gray!5!white}
    
% \usepackage{slashbox}
\setbeamertemplate{navigation symbols}{}%remove navigation symbols
\definecolor{UniBlue}{RGB}{1,1,1}
\setbeamercolor{title}{fg=UniBlue}
\setbeamercolor{frametitle}{fg=UniBlue}
\setbeamercolor{structure}{fg=UniBlue}

\renewcommand*{\thefootnote}{\fnsymbol{footnote}}

\newcommand{\ellcap}{{\ell_\text{c}}}
\newcommand{\F}{{\boldsymbol F}}
\newcommand{\Fh}{{{F}_h}}
%\newcommand{\Fv}{{\boldsymbol{F}_v}}
\newcommand{\Tv}{{\boldsymbol{T}}}
\newcommand{\bn}{{\boldsymbol{\hat n}}}
\newcommand{\bN}{{\boldsymbol{\hat N}}}
\newcommand{\bt}{{\boldsymbol{\hat t}}}
\newcommand{\bz}{{\boldsymbol{z}}}
\newcommand{\bx}{{\boldsymbol{x}}}
\newcommand{\by}{{\boldsymbol{y}}}
\newcommand{\bT}{{\boldsymbol{T}}}
\newcommand{\grad}{\boldsymbol{\nabla}}

\newcommand{\back}[1]{{\color{gray}{#1}}}

\definecolor{darker}{gray}{0.81569}
\definecolor{lighter}{gray}{0.67843}

\title{The structure, function, and evolution of the human foot}
% \subtitle{Transverse arch made foot stiff}
\author{Shreyas Mandre, Brown University}
\date{Dynamic walking, \\ \scriptsize \today }
\begin{document}

\begin{frame}

\centerline{\Large Around a {\bf camphoric-acid boat}, is the surfactant}
~\\[1mm]
\centerline{\Large adsorped on to the interface or dissolved in the bulk?}
% \centerline{\large Stiffness of the human foot and evolution of the transverse arch}
~ \\[1mm]
\begin{columns}
\column{4.25cm}
\centerline{\small Shreyas Mandre}
\centerline{\scriptsize Brown University, Providence RI USA}
\column{4.25cm}
\centerline{\small Mahesh Bandi}
\centerline{\scriptsize OIST, Okinawa Japan}
\end{columns}
% ~\\[1mm]
\begin{center}
\begin{tabular}{ccc}
{\scriptsize Ravi Singh} &
{\scriptsize Dhiraj Singh} &
{\scriptsize Sathish Akella} \\
{\scriptsize Brown University} &
{\scriptsize OIST} &
{\scriptsize OIST}
\end{tabular}
\end{center}
~ \\[1mm]

\centerline{\small APS DFD 2016}
\end{frame}

\begin{frame}{What is a camphoric-acid boat?}

\begin{columns}
\column{6cm}
\only<1>{
\centerline{
\movie[autostart,loop,height=4cm]{\pgfimage[height=4cm]{Figures/sigma72dypcm-001.jpeg}}{Figures/sigma72dypcm.avi}
}
}
\only<2>{
\centerline{
\movie[autostart,loop,height=4cm]{\pgfimage[height=4cm]{Figures/CBconvection-001.jpeg}}{Figures/CBconvection.avi}
}
}
\column{6cm}
\only<2>{
\begin{mdframed}[style=MyFrame]
\scriptsize
{\bf Notable previous work:}
\scriptsize
\begin{itemize}
 \item B. Franklin in a Letter to Dr. Brownrigg, November 7, 1773. Published as Posthumous Writings of Dr. B. Franklin, F.R.S. \&c. London, 1819.
 \item Volta, {\it Delectus Opusculorum Medicorum,} edited by Frank. Ticini, 1787.
 \item Venturi, Annales de Chimie, vol. xxi. p. 262.
 \item Carradori, various publications dated 1794, 1800, 1800, 1803, 1812.
 \item Biot, Bulletin des Sciences par la Societe Philomatique, v54, p42, 1801.
\end{itemize}
\end{mdframed}
}

\only<1>{\pgfimage[width=6.33cm]{Figures/CBoat_Stamp_Schematic}}
\end{columns}

\begin{columns}
\column{2cm}
\centerline{\pgfimage[height=2cm]{Figures/CamphoricAcidMolecule}}
\centerline{\tiny Camphoric acid}
\column{4cm}
\begin{mdframed}[style=MyFrame]
{\scriptsize 
{\bf Properties:} \\[1mm]
\begin{tabular}{ll}
Molecular weight& : 200.23 \\[1mm]
Solubility in H$_2$O& : 8 g/L 
\end{tabular}
}
\end{mdframed}

\column{6cm}
\begin{mdframed}[style=MyFrame]
 {\scriptsize
{\bf Conclusion of this talk:} \\
Marangoni flow  is driven by a layer of camphoric acid \\
adsorbed on the air-water interface.
}
\end{mdframed}
\end{columns}
\end{frame}

\begin{frame}{Approach: $u \sim K r^n$}
\begin{columns}
\column{6cm}
\begin{mdframed}[style=MyFrame]
\centerline{\bf Experiments}
\only<1>{
{\scriptsize
\centerline{\pgfimage[width=6cm]{Figures/Experiment_Schematic}}
}
}
\only<2>{
\centerline{
\movie[autostart,loop,height=3.4cm]{\pgfimage[height=3.4cm]{Figures/radialvelocity-001.jpeg}}{Figures/radialvelocity.avi}
}
}
\end{mdframed}
\column{6cm}
\begin{mdframed}[style=MyFrame]
\centerline{\bf Theory}
{\scriptsize
\centerline{\pgfimage[width=6cm]{Figures/Theory_Schematic}}
}
\end{mdframed}

\end{columns}

\begin{columns}
\column{6cm}
\begin{mdframed}[style=MyFrame]
\centerline{\bf Strategy}
~\\[-5mm]
{\scriptsize
\begin{itemize}
 \item Assume nature of surfactant,
 \item predict axisymmetric fluid velocity,
 \item compare with experiments.
\end{itemize}
}
\end{mdframed}
\column{6cm}
\begin{mdframed}[style=MyFrame]
\centerline{\bf Challenges}
~\\[-5mm]
{\scriptsize
\begin{itemize}
 \item Rate of surfactant release unknown,
 \item Surfactant equation of state unknown,
 \item Singular dependence on viscosity.
\end{itemize}
}
\end{mdframed}
\end{columns}

~\\[3mm]
{\scriptsize {\bf Resolution:} Compare certain features of the flow that are independent of the unknown parameters.}
\end{frame}

\begin{frame}{Theoretical predictions: $u(r) \sim K r^n$}

\begin{mdframed}[style=MyFrame]
{\scriptsize
\begin{align*}
&\text{\bf Momentum conservation}:&  &\rho(uu_r + w u_z) = \mu u_{zz}& &\quad \Longrightarrow \quad&  &\dfrac{\rho u^2}{r} = \mu \dfrac{u}{\delta^2}&. \\
&\text{\bf Marangoni stress}:&       &\mu u_z = \sigma_r&        &\quad \Longrightarrow \quad&  &\dfrac{\mu u}{\delta} = \dfrac{\Delta \sigma}{r}&
\end{align*}
\centerline{Surface tension $\sigma$ depends on the surfactant transport.}
}
\end{mdframed}


\begin{columns}
\column{3.5cm}
\begin{mdframed}[style=MyFrame]
{\scriptsize
\centerline{\bf Adsorbed}
\only<1>{
\vspace{1mm}
$\Delta \sigma = -\Gamma_2 c_2$ \\[1mm]
$u r c_2 = Q_2 = \text{const}$ \\[3mm]
\centerline{$u \sim \boldsymbol{\dfrac{1}{r^{3/5}}} \left(\dfrac{Q_2^2 \Gamma_2^2}{\mu \rho}\right)^{1/5}$}
\vspace{0.5mm}
}
\only<2>{
\centerline{\pgfimage[width=3.5cm]{Figures/Insoluble_scaling}}
}
}
\end{mdframed}

\column{3.5cm}
\begin{mdframed}[style=MyFrame]
{\scriptsize
\centerline{\bf Soluble}
\only<1>{
\vspace{1mm}
$\Delta \sigma = -\Gamma_3 c_3$ \\[1mm]
$u r c_3 \delta = Q_3 = \text{const}$ \\[3mm]
\centerline{$u \sim \boldsymbol{\dfrac{1}{r^{\vphantom{1}}}} \left(\dfrac{Q_3^{\vphantom{2}} \Gamma_3^{\vphantom{2}}}{\mu}\right)^{1/2} = \dfrac{K_3}{r}$}
\vspace{0.5mm}
}
\only<2>{
\centerline{\pgfimage[width=3.5cm]{Figures/Soluble_scaling}}
}
}
\end{mdframed}

\column{3.5cm}
\begin{mdframed}[style=MyFrame]
{\scriptsize
\centerline{\bf Volatile}
\only<1>{
\vspace{5mm}
$\Delta \sigma = -\Gamma_3 c_3$ \\[1mm]
surfactant evaporation rate? \\[3mm]
\centerline{$u \sim \boldsymbol{\dfrac{1}{r^{2/3}}} ? $}
\vspace{0.5mm}
}
\only<2>{
\centerline{\pgfimage[width=3.cm]{Figures/Question_Mark}}
}
}
\end{mdframed}
\end{columns}
~\\[1mm]
{\tiny Close analogy with boundary layers in thermo-marangoni flows (Napolitano, 1979; Zebib, Homsy and Meiberg, 1985; Carpenter and Homsy, 1990) and surfactant transport (Jensen, 1994, dis1995).}
\end{frame}

\begin{frame}{Experimental measurements}

{\scriptsize
\begin{columns}
\column{6cm}
\begin{mdframed}[style=MyFrame]
\centerline{\bf Methanol (volatile)}
\centerline{\pgfimage[width=6cm]{Figures/Methanol_scaling}}
\end{mdframed}

\column{6cm}
\begin{mdframed}[style=MyFrame]
\centerline{\bf Camphoric Acid (unknown)}
\centerline{\pgfimage[width=6cm]{Figures/Cacid_scaling}}
\end{mdframed}
\end{columns}
}
~\\[5mm]
\centerline{Distinction between a power law exponent of $3/5 = 0.6$ and $2/3 \approx 0.667$}
\end{frame}

\begin{frame}{Distinguishing exponents}

\begin{columns}
\column{6cm}
\begin{mdframed}[style=MyFrame]
\centerline{\bf Laser Doppler Velocimetry}
\centerline{\pgfimage[width=5cm]{Figures/steadystate_setup_27june}}
\end{mdframed}

\column{6cm}
\begin{mdframed}[style=MyFrame]
\centerline{\bf Measured exponent}
\centerline{\pgfimage[width=6.2cm]{Figures/Cacid_exponent}}
\end{mdframed}

\end{columns}

\begin{columns}
\column{4cm}
\begin{mdframed}[style=MyFrame]
{\scriptsize
{\bf Measurement details:}
\begin{itemize}
 \item 5-digit accuracy in velocity 
 \item Spatial resolution $200~\mu m$.
 \item Change fluid viscosity by using water mixed with 0\%, 10\%, 20\%, 30\%, 35\% glycerol.
\end{itemize}

}
\end{mdframed}

\column{7cm}
\begin{mdframed}[style=MyFrame]
{\scriptsize 
{\bf Isolating the exponent:}
\begin{align*}
u(r) = K r^n \quad \rightarrow \quad \log u = n \log r + \log K \quad \rightarrow \quad \boldsymbol{n = \dfrac{d (\log u)}{d (\log r)}} 
\end{align*}
\vspace{-6mm}

{\tiny 
$n=-3/5$ ~~: Adsorbed surfactant \\[-1mm]
$n=-2/3?$ : Volatile miscible surfactant \\[-1mm]
$n=-1$ ~~~~~: Soluble/miscible surfactant
}
}
\end{mdframed}
\end{columns}
\end{frame}

\begin{frame}{Conclusion}
\begin{columns}
\column{6cm}
\begin{mdframed}[style=MyFrame]
{\scriptsize 
{\bf Summary of results:} \\
\begin{itemize}
 \item Distinguish between $u \sim r^{-3/5}$ and $r^{-2/3}$.
 \item For camphoric acid boat: $u(r) \sim K r^{-3/5}$.
 \item Therefore, camhoric acid must be spreading as an adsorbed phase, despite being soluble in water.
\end{itemize}
}
\end{mdframed}

\begin{mdframed}[style=MyFrame]
{\scriptsize 
{\bf Ongoing work:} \\
\begin{itemize}
 \item Mechanism for volatile surfactants
\end{itemize}
}


\end{mdframed}
\centerline{
\movie[autostart,loop,height=3.5cm]{\pgfimage[height=3.5cm]{Figures/sigma72dypcm-001.jpeg}}{Figures/sigma72dypcm.avi}
}

\column{6cm}
\begin{mdframed}[style=MyFrame]
{\scriptsize
\centerline{\bf Measured exponent}
\centerline{\pgfimage[width=6.2cm]{Figures/Cacid_exponent}}
}
\end{mdframed}
\begin{mdframed}[style=MyFrame]
{\scriptsize 
{\bf Theoretical predictions: $u(r) = K r^n$} \\
{\scriptsize
$n=-3/5$ ~~: Adsorbed surfactant \\[-1mm]
$n=-2/3?$ : Volatile miscible surfactant \\[-1mm]
$n=-1$ ~~~~~: Soluble/miscible surfactant
}
}
\end{mdframed}
\end{columns}

\end{frame}


\end{document}
